\documentclass[letterpaper]{book}
\usepackage[times,inconsolata,hyper]{Rd}
\usepackage{makeidx}
\usepackage[utf8]{inputenc} % @SET ENCODING@
% \usepackage{graphicx} % @USE GRAPHICX@
\makeindex{}
\begin{document}
\chapter*{}
\begin{center}
{\textbf{\huge Package `covid19br'}}
\par\bigskip{\large \today}
\end{center}
\inputencoding{utf8}
\ifthenelse{\boolean{Rd@use@hyper}}{\hypersetup{pdftitle = {covid19br: Brazilian COVID-19 Pandemic Data}}}{}\ifthenelse{\boolean{Rd@use@hyper}}{\hypersetup{pdfauthor = {Fabio Demarqui; Cristiano Santos}}}{}\begin{description}
\raggedright{}
\item[Type]\AsIs{Package}
\item[Title]\AsIs{Brazilian COVID-19 Pandemic Data}
\item[Version]\AsIs{0.1.3}
\item[Description]\AsIs{Set of functions to import COVID-19 pandemic data into R. The Brazilian COVID-19 data, obtained from the official Brazilian repository at <}\url{https://covid.saude.gov.br/}\AsIs{>, is available at country, region, state, and city-levels. The package also downloads the world-level COVID-19 data from the John Hopkins University's repository.}
\item[URL]\AsIs{}\url{https://fndemarqui.github.io/covid19br/}\AsIs{}
\item[BugReports]\AsIs{}\url{https://github.com/fndemarqui/covid19br/issues}\AsIs{}
\item[Encoding]\AsIs{UTF-8}
\item[License]\AsIs{AGPL (>= 3)}
\item[Depends]\AsIs{R (>= 3.5.0)}
\item[Imports]\AsIs{data.table,
dplyr,
rio,
rlang,
sf,
tidyr}
\item[Suggests]\AsIs{ggrepel,
kableExtra,
knitr,
leaflet,
pracma,
plotly,
rmarkdown,
testthat,
tidyverse}
\item[LazyData]\AsIs{true}
\item[RoxygenNote]\AsIs{7.1.2}
\item[VignetteBuilder]\AsIs{knitr}
\end{description}
\Rdcontents{\R{} topics documented:}
\inputencoding{utf8}
\HeaderA{add\_epi\_rates}{Adding incidence, mortality and lethality rates to the downloaded data}{add.Rul.epi.Rul.rates}
%
\begin{Description}\relax
This function adds the incidence, mortality and lethality rates to a given data set downloaded by the covid19br::downloadCovid19() function.
\end{Description}
%
\begin{Usage}
\begin{verbatim}
add_epi_rates(data, ...)
\end{verbatim}
\end{Usage}
%
\begin{Arguments}
\begin{ldescription}
\item[\code{data}] a data set downloaded using the covid19br::downloadCovid19() function.

\item[\code{...}] further arguments passed to other methods.
\end{ldescription}
\end{Arguments}
%
\begin{Details}\relax
The function add\_epi\_rates() was designed to work with the original names of the variables accumDeaths, accummCases and pop available in the data set downloaded by the covid19br::downloadCovid19(). For this reason, this function must be used before any change in such variable names.
\end{Details}
%
\begin{Value}
the data set with the added incidence, mortality and lethality rates.
\end{Value}
%
\begin{Author}\relax
Fabio N. Demarqui \email{fndemarqui@est.ufmg.br}
\end{Author}
%
\begin{Examples}
\begin{ExampleCode}

library(covid19br)

brazil <- downloadCovid19(level = "brazil")
brazil <- add_epi_rates(brazil)


\end{ExampleCode}
\end{Examples}
\inputencoding{utf8}
\HeaderA{add\_geo}{Adding the geometry to the downloaded data for drawing maps}{add.Rul.geo}
%
\begin{Description}\relax
This function adds the necessary geometry for drawing maps to a given data set downloaded by the covid19br::downloadCovid19() function.
\end{Description}
%
\begin{Usage}
\begin{verbatim}
add_geo(data, ...)
\end{verbatim}
\end{Usage}
%
\begin{Arguments}
\begin{ldescription}
\item[\code{data}] a data set downloaded using the covid19br::downloadCovid19() function.

\item[\code{...}] further arguments passed to other methods.
\end{ldescription}
\end{Arguments}
%
\begin{Details}\relax
The function add\_geo() was designed to work with the original names of the variables available in the data set downloaded by the covid19br::downloadCovid19(). For this reason, this function must be used before any change in such variable names.
\end{Details}
%
\begin{Value}
the data set with the added georeferenced data.
\end{Value}
%
\begin{Author}\relax
Fabio N. Demarqui \email{fndemarqui@est.ufmg.br}
\end{Author}
%
\begin{Examples}
\begin{ExampleCode}

library(covid19br)

regions <- downloadCovid19(level = "regions")
regions_geo <- add_geo(regions)


\end{ExampleCode}
\end{Examples}
\inputencoding{utf8}
\HeaderA{covid19br}{Brazilian COVID-19 Pandemic Data.}{covid19br}
%
\begin{Description}\relax
The package provides a function to automatically import  Brazilian CODID-19 pandemic data into R. Brazilian data is available on the country, region, state, and city levels, and are obtained from the official Brazilian repository at <https://covid.saude.gov.br/>. The package also downloads the world-level COVID-19 data from the John Hopkins University's repository at <https://github.com/CSSEGISandData/COVID-19>.
\end{Description}
%
\begin{Author}\relax
Fábio N. Demarqui, Cristiano C. Santos, and Matheus B. Costa.
\end{Author}
\inputencoding{utf8}
\HeaderA{downloadCovid19}{Function to download COVID-19 data from web repositories}{downloadCovid19}
\aliasA{downloadCovid}{downloadCovid19}{downloadCovid}
%
\begin{Description}\relax
This function downloads the pandemic COVID-19 data at Brazil and World basis. Brazilan data is available at national, region, state, and city levels, whereas the world data are available at the country level.
\end{Description}
%
\begin{Usage}
\begin{verbatim}
downloadCovid19(level = c("brazil", "regions", "states", "cities", "world"))
\end{verbatim}
\end{Usage}
%
\begin{Arguments}
\begin{ldescription}
\item[\code{level}] the desired level of data aggregation:  "brazil" (default), "regions", "states", "cities", and "world".
\end{ldescription}
\end{Arguments}
%
\begin{Details}\relax
The Brazilian data provided by the Brazilian government at its official repository (https://covid.saude.gov.br/) is available in multiple xlsx files. Those files contains data aggregated at national, state, and city geographic levels.  Because importing such data file into R requires a considerable amount of RAM (currently over 4G), the data is daily downloaded and then made available in smaller/lighter binary files on the GitHub repository (https://github.com/dest-ufmg/covid19repo) maintained by the authors' package.
\end{Details}
%
\begin{Value}
a tibble containing the downloaded data at the specified level.
\end{Value}
%
\begin{Examples}
\begin{ExampleCode}

library(covid19br)

# Downloading Brazilian COVID-19 data:
brazil <- downloadCovid19(level = "brazil")
regions <- downloadCovid19(level = "regions")
states <- downloadCovid19(level = "states")
cities <- downloadCovid19(level = "cities")

# Downloading world COVID-19 data:
world <- downloadCovid19(level = "world")


\end{ExampleCode}
\end{Examples}
\inputencoding{utf8}
\HeaderA{election2018Cities}{Results of the 2018 presidential election in Brazil by city.}{election2018Cities}
\keyword{datasets}{election2018Cities}
%
\begin{Description}\relax
Data set containing the results of the 2018 presidential election in Brazil.
\end{Description}
%
\begin{Format}
A data frame with 5570 rows and 6 variables:
\begin{itemize}

\item{} region: regions' names
\item{} state: states' names.
\item{} city: cities' names.
\item{} region\_code: numerical code attributed to regions
\item{} state\_code: numerical code attributed to states
\item{} mesoregion\_code: numerical code attributed to mesoregions
\item{} microregion\_code: numerical code attributed to microregions
\item{} city\_code: numerical code attributed to cities
\item{} Bolsonaro: count of votes obtained by the President-elected Jair Bolosnaro.
\item{} Haddad: count of votes obtained by the defeated candidate Fernando Haddad.
\item{} pop: estimated population in 2019.

\end{itemize}

\end{Format}
%
\begin{Author}\relax
Fabio N. Demarqui \email{fndemarqui@est.ufmg.br}
\end{Author}
%
\begin{Source}\relax
Tribunal Superior Eleitoral (TSE). URL:  \url{https://www.tse.jus.br/eleicoes/estatisticas}.
\end{Source}
\inputencoding{utf8}
\HeaderA{election2018Regions}{Results of the 2018 presidential election in Brazil by region.}{election2018Regions}
\keyword{datasets}{election2018Regions}
%
\begin{Description}\relax
Data set containing the results of the 2018 presidential election in Brazil.
\end{Description}
%
\begin{Format}
A data frame with 5 rows and 4 variables:
\begin{itemize}

\item{} region: regions' names.
\item{} Bolsonaro: count of votes obtained by the President-elected Jair Bolosnaro.
\item{} Haddad: count of votes obtained by the defeated candidate Fernando Haddad.
\item{} pop: estimated population in 2019.

\end{itemize}

\end{Format}
%
\begin{Author}\relax
Fabio N. Demarqui \email{fndemarqui@est.ufmg.br}
\end{Author}
%
\begin{Source}\relax
Tribunal Superior Eleitoral (TSE). URL:  \url{https://www.tse.jus.br/eleicoes/estatisticas}.
\end{Source}
\inputencoding{utf8}
\HeaderA{election2018States}{Results of the 2018 presidential election in Brazil by state.}{election2018States}
\keyword{datasets}{election2018States}
%
\begin{Description}\relax
Data set containing the results of the 2018 presidential election in Brazil.
\end{Description}
%
\begin{Format}
A data frame with 27 rows and 5 variables:
\begin{itemize}

\item{} region: regions' names.
\item{} state: states' names.
\item{} Bolsonaro: count of votes obtained by the President-elected Jair Bolosnaro.
\item{} Haddad: count of votes obtained by the defeated candidate Fernando Haddad.
\item{} pop: estimated population in 2019.

\end{itemize}

\end{Format}
%
\begin{Author}\relax
Fabio N. Demarqui \email{fndemarqui@est.ufmg.br}
\end{Author}
%
\begin{Source}\relax
Tribunal Superior Eleitoral (TSE). URL:  \url{https://www.tse.jus.br/eleicoes/estatisticas}.
\end{Source}
\inputencoding{utf8}
\HeaderA{ibgeCities}{City-level georeferenced data}{ibgeCities}
\keyword{datasets}{ibgeCities}
%
\begin{Description}\relax
Data set obtaind from the Instituto Brasileiro de Geografia e Estatística (IBGE) with data on the Brazilian population and geographical information on city level.
\end{Description}
%
\begin{Format}
A data frame with 5570 rows and 10 variables:
\begin{itemize}

\item{} region: regions' names
\item{} state: states' names.
\item{} city: cities' names.
\item{} pop: estimated population in 2019.
\item{} region\_code: numerical code attributed to regions
\item{} state\_code: numerical code attributed to states
\item{} mesoregion\_code: numerical code attributed to mesoregions
\item{} microregion\_code: numerical code attributed to microregions
\item{} city\_code: numerical code attributed to cities
\item{} geometry: georeferenced data needed to plot maps.
\item{} area: area (in Km\textasciicircum{}2) of the brazilian cities
\item{} demoDens: demographic density of the brazilian cities.

\end{itemize}

\end{Format}
%
\begin{Author}\relax
Fabio N. Demarqui \email{fndemarqui@est.ufmg.br}
\end{Author}
%
\begin{Source}\relax
Instituto Brasileiro de Geografia e Estatística (IBGE):
\begin{itemize}

\item{} Shapefiles: \url{https://www.ibge.gov.br/geociencias/downloads-geociencias.html}
\item{} Population: \url{https://www.ibge.gov.br/estatisticas/sociais/populacao/9103-estimativas-de-populacao.html?=&t=resultados}

\end{itemize}

\end{Source}
\inputencoding{utf8}
\HeaderA{ibgeRegions}{Region-level georeferenced data}{ibgeRegions}
\keyword{datasets}{ibgeRegions}
%
\begin{Description}\relax
Data set obtaind from the Instituto Brasileiro de Geografia e Estatística (IBGE) with data on the Brazilian population and geographical information on region level.
\end{Description}
%
\begin{Format}
A data frame with 5 rows and 4 variables:
\begin{itemize}

\item{} region: regions' names
\item{} pop: estimated population in 2019.
\item{} region\_code: numerical code attributed to regions
\item{} geometry: georeferenced data needed to plot maps.
\item{} area: area (in Km\textasciicircum{}2) of the brazilian regions.
\item{} demoDens: demographic density of the brazilian regions.

\end{itemize}

\end{Format}
%
\begin{Author}\relax
Fabio N. Demarqui \email{fndemarqui@est.ufmg.br}
\end{Author}
%
\begin{Source}\relax
Instituto Brasileiro de Geografia e Estatística (IBGE):
\begin{itemize}

\item{} Shapefiles: \url{https://www.ibge.gov.br/geociencias/downloads-geociencias.html}
\item{} Population: \url{https://www.ibge.gov.br/estatisticas/sociais/populacao/9103-estimativas-de-populacao.html?=&t=resultados}

\end{itemize}

\end{Source}
\inputencoding{utf8}
\HeaderA{ibgeStates}{State-level georeferenced data}{ibgeStates}
\keyword{datasets}{ibgeStates}
%
\begin{Description}\relax
Data set obtaind from the Instituto Brasileiro de Geografia e Estatística (IBGE) with data on the Brazilian population and geographical information on state level.
\end{Description}
%
\begin{Format}
A data frame with 27 rows and 6 variables:
\begin{itemize}

\item{} region: regions' names
\item{} state: states' names.
\item{} pop: estimated population in 2019.
\item{} region\_code: numerical code attributed to regions
\item{} state\_code: numerical code attributed to states
\item{} geometry: georeferenced data needed to plot maps.
\item{} area: area (in Km\textasciicircum{}2) of the brazilian states.
\item{} demoDens: demographic density of the brazilian states.

\end{itemize}

\end{Format}
%
\begin{Author}\relax
Fabio N. Demarqui \email{fndemarqui@est.ufmg.br}
\end{Author}
%
\begin{Source}\relax
Instituto Brasileiro de Geografia e Estatística (IBGE):
\begin{itemize}

\item{} Shapefiles: \url{https://www.ibge.gov.br/geociencias/downloads-geociencias.html}
\item{} Population: \url{https://www.ibge.gov.br/estatisticas/sociais/populacao/9103-estimativas-de-populacao.html?=&t=resultados}

\end{itemize}

\end{Source}
\inputencoding{utf8}
\HeaderA{ipeaCities}{Brazilian development human indexes by cities}{ipeaCities}
\keyword{datasets}{ipeaCities}
%
\begin{Description}\relax
Data set on the development humam indexes provided the Instituto de Pesquisa Econômica Aplicada in 2010.
\end{Description}
%
\begin{Format}
A data frame with 5570 rows and 9 variables:
\begin{itemize}

\item{} region: regions' names.
\item{} state: states' names.
\item{} city: states' names.
\item{} city\_code: numerical code attributed to cities
\item{} DHI: development human index.
\item{} EDHI: educational development human index.
\item{} LDHI: longevity development human index.
\item{} IDHI: income development human index.
\item{} pop: estimated population in 2019.

\end{itemize}

\end{Format}
%
\begin{Author}\relax
Fabio N. Demarqui \email{fndemarqui@est.ufmg.br}
\end{Author}
%
\begin{Source}\relax
Instituto de Pesquisa Econômica Aplicada (IPEA). URL: \url{https://www.ipea.gov.br/ipeageo/bases.html}.
\end{Source}
\inputencoding{utf8}
\HeaderA{ipeaRegions}{Brazilian development human indexes by regions}{ipeaRegions}
\keyword{datasets}{ipeaRegions}
%
\begin{Description}\relax
Data set on the development humam indexes provided the Instituto de Pesquisa Econômica Aplicada in 2010.
\end{Description}
%
\begin{Format}
A data frame with 5 rows and 6 variables:
\begin{itemize}

\item{} region: regions' names.
\item{} DHI: development human index.
\item{} EDHI: educational development human index.
\item{} LDHI: longevity development human index.
\item{} IDHI: income development human index.
\item{} pop: estimated population in 2019.

\end{itemize}

\end{Format}
%
\begin{Author}\relax
Fabio N. Demarqui \email{fndemarqui@est.ufmg.br}
\end{Author}
%
\begin{Source}\relax
Instituto de Pesquisa Econômica Aplicada (IPEA). URL: \url{https://www.ipea.gov.br/ipeageo/bases.html}.
\end{Source}
\inputencoding{utf8}
\HeaderA{ipeaStates}{Brazilian development human indexes by states}{ipeaStates}
\keyword{datasets}{ipeaStates}
%
\begin{Description}\relax
Data set on the development humam indexes provided the Instituto de Pesquisa Econômica Aplicada in 2010.
\end{Description}
%
\begin{Format}
A data frame with 27 rows and 6 variables:
\begin{itemize}

\item{} region: regions' names.
\item{} state: states' names.
\item{} DHI: development human index.
\item{} EDHI: educational development human index.
\item{} LDHI: longevity development human index.
\item{} IDHI: income development human index.
\item{} pop: estimated population in 2019.

\end{itemize}

\end{Format}
%
\begin{Author}\relax
Fabio N. Demarqui \email{fndemarqui@est.ufmg.br}
\end{Author}
%
\begin{Source}\relax
Instituto de Pesquisa Econômica Aplicada (IPEA). URL: \url{https://www.ipea.gov.br/ipeageo/bases.html}.
\end{Source}
\inputencoding{utf8}
\HeaderA{mundi}{World-level georeferenced data}{mundi}
\keyword{datasets}{mundi}
%
\begin{Description}\relax
Data set extracted from the R package rnaturalearthdata.
\end{Description}
%
\begin{Format}
A data frame with 241 rows and 12 variables:
\begin{itemize}

\item{} country: country's name
\item{} continent: continent's name
\item{} region: regions' names
\item{} subregion: subregion's name
\item{} pop: estimated population
\item{} pais: country's name in Portuguese
\item{} country\_code: numerical code attributed to countries
\item{} continent\_code: numerical code attributed to continents
\item{} region\_code: numerical code attributed to regions
\item{} subregion\_code: numerical code attributed to subregions
\item{} geometry: georeferenced data needed to plot maps.

\end{itemize}

\end{Format}
%
\begin{Author}\relax
Fabio N. Demarqui \email{fndemarqui@est.ufmg.br}
\end{Author}
%
\begin{Source}\relax
R package rnaturalearthdata.
\end{Source}
\printindex{}
\end{document}
